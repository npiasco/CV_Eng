%!TEX TS-program = xelatex
\documentclass[]{friggeri-cv-cust}
\usepackage{enumitem}
\usepackage{amssymb}
\usepackage{afterpage}
\usepackage[colorlinks=true]{hyperref}
\usepackage{color}
\usepackage{xcolor}
\hypersetup{
    pdftitle={},
    pdfauthor={},
    pdfsubject={},
    pdfkeywords={},
    colorlinks = true,
	linkcolor = blue,
	urlcolor  = blue,
    citecolor = blue,
    anchorcolor = blue       % no lik border color
   allbordercolors=white    % white border color for all
}
\addbibresource{bibliography.bib}
%\RequirePackage{xcolor}
\color{darkgray}
\begin{document}

\header{Nathan}{Piasco}
      {Ingénieur - Doctorant en vision par ordinateur}
      
% Fake text to add separator      
\vspace{10pt}
\fcolorbox{white}{gray}{\parbox{\dimexpr\textwidth-2\fboxsep-2\fboxrule}{%
.....
}}
\vspace{1pt}

% In the aside, each new line forces a line break
\begin{aside}
  ~
  \section{Contact}
    ~
    10 rue des Arquebusiers
    75003, Paris
    FRANCE
    ~
    +33 632508367
    ~
   	\href{mailto:nathan.piasco@gmail.com}{\textbf{nathan.piasco@}}
   	gmail.com
	{\href{https://www.linkedin.com/in/nathan-piasco-a4766aa5/}{\textcolor{highlight}{linkedin}}}
	{\href{https://scholar.google.fr/citations?user=S3zYmOYAAAAJ&hl=fr}{\textcolor{highlight}{scholar}}}
	~
    Né le 12 août 1992
    ~    
  \section{Expertise}
    ~
    \textbf{\small{\textsc{Vision par ordinateur}}}
    indexation d'images
    géometrie multi-vues
    reconnaissance de formes
    ~    
    \textbf{\small{\textsc{Apprentissage}}}
    supervisé
    non-supervisé
    auto-supervisé
    regression
    ~    
    \textbf{\small{\textsc{Réseau neuronaux}}}
    réseau convolutif
    //~~~~~~ récurrent
    encoder/decoder
    GAN
    ~    
  \section{Informatique}
    ~  
    \textbf{\textsc{\textsc{Langages de Programmation}}}
    Python, \texttt{C/C++},
    Matlab,  \LaTeX
    ~
    \textbf{\textsc{Bibliothèques}}
    Pytorch
    Scikit Learn
	OpenCV, OpenGL
    PCL (Point Cloud Library)
    ROS (Robotic Operating System)
    ~
    \textbf{\textsc{OS}}
    Linux, Windows, MacOS
    ~
    \textbf{{Github}} : {\href{https://github.com/npiasco}{\textcolor{highlight}{npiasco}}}
\end{aside}

\section{Formation}
\begin{entrylist}
  \entry
    {2016 - 2019}
    {Doctorat - Traitement d'Images}
    {UBFC, Dijon}
    {Sujet : Localisation basée vision à partir de caractéristiques discriminantes issues de données visuelles hétérogènes}
  \entry
    {2014 - 2015}
    {Master 2}
    {Universit\'e Pierre et Marie Curie, Paris}
    {Sciences de l'Ingénieur spécialité Image et Son pour les Systèmes Intelligents}
  \entry
    {2010 - 2015}
    {École d’ingénieurs diplômés}
    {Polytech Paris-UPMC, Paris}
    {Spécialité Robotique}
\end{entrylist}

\section{Expériences professionnelles}
\begin{entrylist}
  \entry
    {\small{10/16 - \textit{09/19}}}
    {Doctorant}
    {Vibot, ImVIA - LASTIG-IGN lab, 94160 Saint-Mandé}
    {\textsc{Localisation basée vision à partir de caractéristiques discriminantes issues de données visuelles hétérogènes :}
    \begin{itemize}[label=$\rhd$]
    	\item Réalisation d'un état de l'art exhaustif sur la problématique de la localisation basée vision,
        \item Utilisation d’algorithmes d'apprentissage profond pour la tache de localisation à partir de données de natures hétérogènes et en conditions difficiles,
        \item Mise au point d'une nouvelle méthode d'estimation de pose précise basée sur une approche géométrique combinée à une représentation apprise de la donnée.
    \end{itemize}
    %\vspace{5pt}
    }
  \entry
    {\small{10/16 - \textit{09/19}}}
    {Enseignant}
    {Ecole d'ingénieur ENSG, 77420 Champs-sur-Marne}
    {\textsc{En parallèle de mon travail de thèse, je dispense des cours à des élèves de l'école d’ingénieurs ENSG :}
    \begin{itemize}[label=$\rhd$]
        \item Réalité augmentée : introduction aux bases de la réalité augmentée, du traitement d'images à l'augmentation 3D de l'environnement
        \item OpenGL : introduction à la programmation graphique
   		\item Python : base de la programmation avec python
    \end{itemize}
    %\vspace{5pt}
    }
  \entry
    {10/15 - 09/16}
    {Ingénieur R\&D vision}
    {A.I.Mergence : startup spécialisée en robotique, 75013 Paris}
    {\textsc{Responsable vision par ordinateur pour un projet de robot mobile de gardiennage pour particulier:}
    \begin{itemize}[label=$\rhd$]
    	\item Veille technologique sur différents domaines de la vision par ordinateur : navigation, reconnaissance de personnes, suivi d'objets dans une séquence vidéo, vision multicaméras, caméras de profondeur,
    	\item Développement de modules pour la stéréo-vision, la détection et la localisation d'éléments sémantiques dans un logis,
        \item Optimisation et intégration d'algorithmes sur une architecture embarquée ARM.
    \end{itemize} \vspace{5pt}}
  \entry
    {03/15 - 09/15}
    {Stage de fin d'études}
    {ONERA, the French Aerospace Lab, 91120 Palaiseau}
    {\textsc{Localisation globale coopérative d’un essaim de drones quadri-rotors à base de vision :}
    \begin{itemize}[label=$\rhd$]
    	\item Estimation de position par stéréo-vision multi-portée,
    	\item Fusion de données inertielles/vision,
        \item Mise en place d’une loi de commande d’un essaim de drones, 
        \item Expérimentation en conditions réelles sur une flotte de trois véhicules.
    \end{itemize} \vspace{5pt}}
\end{entrylist}

\newpage
\section{Publications}
    \textsc{\textbf{Peer-reviewed Journal}}
    
    {\footnotesize{N. Piasco, D. Sidibé, C. Demonceaux, V. Gouet-Brunet, \href{http://recherche.ign.fr/labos/matis/pdf/articles_revues/2017/PSDG17.pdf}{\textcolor{highlight}{A Survey on Visual-Based Localization: On the Benefit of Heterogeneous Data}},
\textit{Pattern Recognition, Volume 74, February,} 2018.}}
	
	\vspace{0.5cm}
    \textsc{\textbf{Peer-reviewed International Conferences}}
   
    {\footnotesize{N. Piasco, D. Sidibé, C. Demonceaux, V. Gouet-Brunet, \href{https://hal.archives-ouvertes.fr/hal-02190840}{\textcolor{highlight}{Perspective-n-Learned-Point: Pose Estimation from Relative Depth}},
    \textit{British Machine Vision Conference, Cardiff, United Kingdom,} 2019. \textbf{Presentation "Spotlight".}}}
        
    
    {\footnotesize{N. Piasco, D. Sidibé, V. Gouet-Brunet, C. Demonceaux, \href{http://recherche.ign.fr/labos/matis/pdf/articles_conf/2019/root.pdf}{\textcolor{highlight}{Learning Scene Geometry for Visual Localization in Challenging Conditions}},
\textit{IEEE International Conference of Robotics and Automation, Montreal, Canada,} 2019. \textbf{Nominé pour le prix du meilleur papier en Vision Robotique.}}}

    {\footnotesize{N. Piasco, D. Sidibé,  C. Demonceaux, V. Gouet-Brunet, \href{https://hal.archives-ouvertes.fr/hal-02123899}{\textcolor{highlight}{Geometric Camera Pose Refinement with Learned Depth Maps}},
\textit{IEEE International Conference on Image Processing, Taipei, Taiwan} 2019.}}

	{\footnotesize{N. Piasco, J. Marzat, M. Sanfourche, \href{http://julien.marzat.free.fr/Publications/2016\%20ICRA/2016_ICRA_Collaborative_localization_formation_flying_distributed_stereo-vision.pdf}{\textcolor{highlight}{Collaborative localization and formation flying using distributed stereo-vision}},
\textit{IEEE International Conference on Robotics and Automation, Stockholm, Sweden,} 2016.}}	

	\vspace{0.5cm}
    \textsc{\textbf{Peer-reviewed National Conferences}}
    
    {\footnotesize{N. Piasco, D. Sidibé, V. Gouet-Brunet, C. Demonceaux, \href{https://rfiap2018.ign.fr/sites/default/files/ARTICLES/RFIAP_2018/RFIAP_2018_Piasco_Apprentissage.pdf}{\textcolor{highlight}{Apprentissage de modalités auxiliaires pour la localisation basée vision}},
\textit{Reconnaissance des Formes, Image, Apprentissage et Perception (RFIAP), Champs-sur-Marne, France,} 2018.}}

\footnotesize{N. Piasco, D. Sidibé, V. Gouet-Brunet, C. Demonceaux, \href{http://recherche.ign.fr/labos/matis/img/ic_pdf.gif}{\textcolor{highlight}{Localisation Basée Vision : de l’hétérogénéité des approches et des données}},
\textit{ORASIS - Journées francophones des jeunes chercheurs en vision par ordinateur, Colleville-sur-Mer, France,} 2017.}
    

%\begin{flushleft}
%\emph{January 14th, 2014}
%\end{flushleft}
%\begin{flushright}
%\emph{Carmine Benedetto}
%\end{flushright}

%%% This piece of code has been commented by Karol Kozioł due to biblatex errors. 
% 
%\printbibsection{article}{article in peer-reviewed journal}
%\begin{refsection}
%  \nocite{*}
%  \printbibliography[sorting=chronological, type=inproceedings, title={international peer-reviewed conferences/proceedings}, notkeyword={france}, heading=subbibliography]
%\end{refsection}
%\begin{refsection}
%  \nocite{*}
%  \printbibliography[sorting=chronological, type=inproceedings, title={local peer-reviewed conferences/proceedings}, keyword={france}, heading=subbibliography]
%\end{refsection}
%\printbibsection{misc}{other publications}
%\printbibsection{report}{research reports}

\begin{asidep2}
  \section{Langues}
    ~
    \textbf{\textsc{Anglais}}
    Courant
    ~
    \textbf{\textsc{Espagnol}}
    Débutant 
    ~    
  \section{Mobilité}
    ~
    Détenteur du permis B
    ~    
  \section{Centres d'intérêts}
    ~
  Réalité augmentée
  Robotique
  Véhicule autonome
\end{asidep2}

\end{document}

%!TEX TS-program = xelatex
\documentclass[]{friggeri-cv-cust}
\usepackage{enumitem}
\usepackage{amssymb}
\usepackage{afterpage}
\usepackage[colorlinks=true]{hyperref}
\usepackage{color}
\usepackage{xcolor}
\hypersetup{
    pdftitle={},
    pdfauthor={},
    pdfsubject={},
    pdfkeywords={},
    colorlinks = true,
	linkcolor = blue,
	urlcolor  = blue,
    citecolor = blue,
    anchorcolor = blue       % no lik border color
   allbordercolors=white    % white border color for all
}
\addbibresource{bibliography.bib}
%\RequirePackage{xcolor}
\color{darkgray}
\begin{document}

\header{Nathan}{Piasco}
      {Engineer - PhD student in computer vision}
      
% Fake text to add separator      
\vspace{10pt}
\fcolorbox{white}{gray}{\parbox{\dimexpr\textwidth-2\fboxsep-2\fboxrule}{%
.....
}}
\vspace{1pt}

% In the aside, each new line forces a line break
\begin{aside}
  ~
  \section{Contact}
    10 rue des Arquebusiers
    75003, Paris
    FRANCE
    ~
    +33 632508367
    ~
   	\href{mailto:nathan.piasco@gmail.com}{\textbf{nathan.piasco@}}
   	gmail.com
	{\href{https://www.linkedin.com/in/nathan-piasco-a4766aa5/}{\textcolor{highlight}{linkedin}}}
	{\href{https://scholar.google.fr/citations?user=S3zYmOYAAAAJ&hl=fr}{\textcolor{highlight}{scholar}}}
	~
    DOB: 12/08/1992
  \section{Expert skills}
    \textbf{\small{\textsc{Computer vision}}}
    image retrieval
    multiple view geometry
    pattern recognition
    \textbf{\small{\textsc{Machine learning}}}
    supervised learning
    unsupervised learning
    selfsupervised learning
    regression
    \textbf{\small{\textsc{Neural networks}}}
    convolutional layer
    recurrent layer
    encoder/decoder
    GAN
  \section{Computer skills}
    \textbf{\textsc{\textsc{Programming}}}
    Python, \texttt{C/C++},
    Matlab,  \LaTeX
    ~
    \textbf{\textsc{Libraries}}
    Pytorch
    Scikit Learn
	OpenCV, OpenGL
    PCL (Point Cloud Library)
    ROS (Robotic Operating System)
    ~
    \textbf{\textsc{OS}}
    Linux, Windows, MacOS
    ~
    \textbf{{Github}} : {\href{https://github.com/npiasco}{\textcolor{highlight}{npiasco}}}
\end{aside}

\section{Education}
\begin{entrylist}
  \entry
    {2016 - 2019}
    {PhD - Image processing}
    {UBFC, Dijon}
    {Thesis : Vision-based localization with discriminative features from heterogeneous visual data}
  \entry
    {2014 - 2015}
    {Master Degree}
    {Universit\'e Pierre et Marie Curie, Paris}
    {Images and Sound processing for Intelligent Systems}
  \entry
    {2010 - 2015}
    {Engineering school}
    {Polytech Paris-UPMC, Paris}
    {Major in Robotic and Computer Programming}
\end{entrylist}

\section{Work experience}
\begin{entrylist}
  \entry
    {\small{10/16 - \textit{09/19}}}
    {PhD student}
    {Vibot, ImVIA - LASTIG-IGN lab, 94160 Saint-Mandé}
    {\textsc{Vision-based localization with discriminative features from heterogeneous visual data:}
    \begin{itemize}[label=$\rhd$]
    	\item Writing of a large review of state-of-the-art methods for visual based localisation with heterogeneous data
        \item Designing of a new deep learned image descriptor for urban image retrieval in challenging condition by exploiting side geometric information
        \item Implementation of a innovative image pose refinement method based on both learned representation and geometric algorithms
    \end{itemize}
    %\vspace{5pt}
    }
  \entry
    {\small{10/16 - \textit{09/19}}}
    {Assistant professor}
    {ENSG engineering school, 77420 Champs-sur-Marne}
    {\textsc{Assistant professor for the following courses:}
    \begin{itemize}[label=$\rhd$]
        \item Practical introduction to augmented reality (master students)
        \item Computer graphics with OpenGL (master students)
   		\item Introduction to object-oriented programming with python (master students)
        \item Programming in python (bachelor \& master students)
    \end{itemize}
    %\vspace{5pt}
    }
  \entry
    {\small{10/15 - 09/16}}
    {Research engineer}
    {A.I.Mergence: robotic startup, 75013 Paris}
    {\textsc{Computer vision referrer in a project of home-safety robot:}
    \begin{itemize}[label=$\rhd$]
    	\item Technological watch on various computer vision related field, including: mapping, people recognition, tracking, multiple view imaging, depth camera
    	\item Development of a stereo camera system for semantic object detection in a house
        \item Optimization and integration of vision algorithms on an embedded ARM device
    \end{itemize}\vspace{5pt}}
  \entry
    {03/15 - 09/15}
    {Master internship}
    {ONERA, the French Aerospace Lab, 91120 Palaiseau}
    {\textsc{Collaborative localization and formation flying using distributed stereo-vision:}
    \begin{itemize}[label=$\rhd$]
    	\item Relative pose estimation of in-flight UAV with a dynamic wide-baseline multiple view system
    	\item Multi-modal and multi-sensor data fusion
        \item Introducing of a new control law for positioning of an UAV swarm 
        \item Practical testing of the developed method in real condition
    \end{itemize} \vspace{5pt}}
\end{entrylist}

\newpage
\section{Publications}
    \textsc{\textbf{Peer-reviewed Journal}}
    
    {\footnotesize{N. Piasco, D. Sidibé, C. Demonceaux, V. Gouet-Brunet, \href{http://recherche.ign.fr/labos/matis/pdf/articles_revues/2017/PSDG17.pdf}{\textcolor{highlight}{A Survey on Visual-Based Localization: On the Benefit of Heterogeneous Data}},
\textit{Pattern Recognition, Volume 74, February,} 2018.}}
	
	\vspace{0.5cm}
    \textsc{\textbf{Peer-reviewed International Conferences}}
   
    {\footnotesize{N. Piasco, D. Sidibé, C. Demonceaux, V. Gouet-Brunet, \href{}{\textcolor{highlight}{Perspective-n-Learned-Point: Pose Estimation from Relative Depth}},
    \textit{British Machine Vision Conference, Cardiff, United Kingdom,} 2019. \textbf{Spotlight presentation.}}}
        
    
    {\footnotesize{N. Piasco, D. Sidibé, V. Gouet-Brunet, C. Demonceaux, \href{http://recherche.ign.fr/labos/matis/pdf/articles_conf/2019/root.pdf}{\textcolor{highlight}{Learning Scene Geometry for Visual Localization in Challenging Conditions}},
\textit{IEEE International Conference of Robotics and Automation, Montreal, Canada,} 2019. \textbf{Finalist nominated for the Best Paper Award in Robot Vision.}}}

    {\footnotesize{N. Piasco, D. Sidibé,  C. Demonceaux, V. Gouet-Brunet, \href{}{\textcolor{highlight}{Geometric Camera Pose Refinement with Learned Depth Maps}},
\textit{IEEE International Conference on Image Processing, Taipei, Taiwan} 2019.}}

	{\footnotesize{N. Piasco, J. Marzat, M. Sanfourche, \href{http://julien.marzat.free.fr/Publications/2016\%20ICRA/2016_ICRA_Collaborative_localization_formation_flying_distributed_stereo-vision.pdf}{\textcolor{highlight}{Collaborative localization and formation flying using distributed stereo-vision}},
\textit{IEEE International Conference on Robotics and Automation, Stockholm, Sweden,} 2016.}}	

	\vspace{0.5cm}
    \textsc{\textbf{Peer-reviewed National Conferences}}
    
    {\footnotesize{N. Piasco, D. Sidibé, V. Gouet-Brunet, C. Demonceaux, \href{https://rfiap2018.ign.fr/sites/default/files/ARTICLES/RFIAP_2018/RFIAP_2018_Piasco_Apprentissage.pdf}{\textcolor{highlight}{Apprentissage de modalités auxiliaires pour la localisation basée vision}},
\textit{Reconnaissance des Formes, Image, Apprentissage et Perception (RFIAP), Champs-sur-Marne, France,} 2018.}}

\footnotesize{N. Piasco, D. Sidibé, V. Gouet-Brunet, C. Demonceaux, \href{http://recherche.ign.fr/labos/matis/img/ic_pdf.gif}{\textcolor{highlight}{Localisation Basée Vision : de l’hétérogénéité des approches et des données}},
\textit{ORASIS - Journées francophones des jeunes chercheurs en vision par ordinateur, Colleville-sur-Mer, France,} 2017.}
    

%\begin{flushleft}
%\emph{January 14th, 2014}
%\end{flushleft}
%\begin{flushright}
%\emph{Carmine Benedetto}
%\end{flushright}

%%% This piece of code has been commented by Karol Kozioł due to biblatex errors. 
% 
%\printbibsection{article}{article in peer-reviewed journal}
%\begin{refsection}
%  \nocite{*}
%  \printbibliography[sorting=chronological, type=inproceedings, title={international peer-reviewed conferences/proceedings}, notkeyword={france}, heading=subbibliography]
%\end{refsection}
%\begin{refsection}
%  \nocite{*}
%  \printbibliography[sorting=chronological, type=inproceedings, title={local peer-reviewed conferences/proceedings}, keyword={france}, heading=subbibliography]
%\end{refsection}
%\printbibsection{misc}{other publications}
%\printbibsection{report}{research reports}

\begin{asidep2}
  \section{Languages}
    \textbf{\textsc{French}}
    Mother tongue
    \textbf{\textsc{English}}
    Fluent
    %~
    \textbf{\textsc{Spanish}}
    Beginner  
  \section{Driving}
    Driver's license holder
  \section{Interests}
  Robotic
  Autonomous driving    
  Augmented reality  
\end{asidep2}

\end{document}
